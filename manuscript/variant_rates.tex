\documentclass[aps,rmp, twocolumn]{revtex4}
%\documentclass[a4paper,10pt]{scrartcl}
%\documentclass[aps,rmp,twocolumn]{revtex4}

\usepackage[utf8]{inputenc}
\usepackage{amsmath,graphicx}
\usepackage{color}
\usepackage{booktabs}
%\usepackage{cite}

\newcommand{\bq}{\begin{equation}}
\newcommand{\eq}{\end{equation}}
\newcommand{\bn}{\begin{eqnarray}}
\newcommand{\en}{\end{eqnarray}}
\newcommand{\Richard}[1]{{\color{red}Richard: #1}}
\newcommand{\gene}[1]{{\it #1}}
\newcommand{\mat}[1]{{\bf #1}}
\newcommand{\vecb}[1]{{\bf #1}}
\newcommand{\abet}{\mathcal{A}}
\newcommand{\eqp}{p}
\newcommand{\LH}{\mathcal{L}}


\begin{document}
\title{Evolutionary rates of SARS-CoV-2}
\author{Richard A.~Neher}
\date{\today}
\maketitle

Since its emergence in late 2019, SARS-CoV-2 has displayed a discontinuous pattern of evolution with large jumps in sequence space giving rise to phylogenetically distinct variants without observed intermediates \citep{hodcroft_spread_2021,volz_assessing_2021,tegally_detection_2021,faria_genomics_2021,naveca_covid-19_2021,viana_rapid_2022}.
Many of these variants spread considerably faster and quickly displaced the resident variants either because of intrinsically increased transmissibility, evasion of prior immunity in the population, or a combination of both.
Such variants of Concern or Interest were designated by the WHO and labeled by Greek letters \citep{konings_sars-cov-2_2021}.
The branches leading to these variants are characterized by many amino acid changing mutations that often cluster in the S1 domain of the spike protein \citep{kistler_rapid_2022}.

This pattern of rapid non-synonymous evolution in viral surface proteins that interact with the host cells in common among many RNA viruses and for example well studied in influenza virus A evolution \citep{bhatt_genomic_2011,strelkowa_clonal_2012}.
Adaptive evolution in influenza tends to be gradual without large jumps in sequence space, while new variants of SARS-CoV-2 with 10s of previously unobserved mutations emerged suddenly without intermediate genomes being observed, the most dramatic being the emergence of Omicron in late 2021 \citep{viana_rapid_2022}.
The pre-dominant hypothesis for this cryptic emergence of highly mutated, transmissible and immune evasive variants are chronic infections that are frequently observed in patients with impaired immune systems, either through HIV-1 infection \citep{cele_sars-cov-2_2022} or medical intervention \citep{choi_persistence_2020,kemp_sars-cov-2_2021}.
Onward transmission from such chronic infections has also been documented \citep{gonzalez-reiche_intrahost_2022}.
However, to date there is no direct evidence for the mode of emergence of any variant.
The case for chronic infection being an important contributor is strongest for the variants Alpha and Omicron \citep{hill_origins_2022}.

The saltatory nature of SARS-CoV-2 evolution implies that evolution within individual variants, clades, or lineages might have different characteristics from its overall pattern, which is dominated by atypical bursts of evolution leading to new variants.
\citet{tay_emergence_2022} and \citet{hill_origins_2022} show that the rate of evolution along these inter-clade branches is up to four-fold higher than the background rate.
Here, we build on these results and investigate the patterns of SARS-CoV-2 diversifications within variants and compare these to the global dynamics of evolution and adaptation.
This comparison reveals a consistent dichotomy between slow within-variant evolution and rapid adaptive evolution giving rise to new variants.
The rate of synonymous evolution does not differ between and within variants, while the non-synonymous rate within variants is far below the between variant rate.
Early variants display more rapid non-synonymous evolution that later variants.
We further quantify the level of functional constraint of different open reading frames and infer a map of mutational tolerance across the genome.

\section*{Results}

Traditionally, evolutionary rates and divergence times are estimates using phylogenetic frameworks for heterochronously sampled sequences.
These methods, however, can not handle the volume of SARS-CoV-2 data available and data have to be dramatically down-sampled.
Furthermore, phylogenetic methods impose an hierarchical structure on the data and are thus very sensitive to problematic sequences or metadata: Any problematic sequence can fatally distort the analysis, in particular if placed early in tree.

To circumvent many of the above mentioned problems and still use the majority of the available data, we use a combination of automated filtering and simple robust approaches to analyze the evolutionary patterns.
We first use Nextclade \citep{aksamentov_nextclade_2021} to assign all sequences to one of the Nextstrain clades \citep{hadfield_nextstrain_2018} and analyze each clade separately.
Sequences belonging to recombinant lineages are excluded.
These clades represent well defined groups of sequences with little non-vertical evolution within them.
Furthermore, we define a ``founder'' genotype for each clade and exclude any sequence that does not have the full set of mutations relative to the reference sequence Wuhan/Hu-1.
This founder sequence is manually curated for each clade considered.
This filtering removes most incomplete sequences as well as sequences where amplicon drop-out are back-filled with reference sequence.
For this reduced set of sequences, we determine the mutations they carry on top of the founder genotype to analyze diversification and divergence within the clade.
The latter step is done for nucleotide changes as well as for amino acid changes.

Within each clade, the number of mutations is expected to increase linearly in time and the variation around this mean would, in an ideal case, obey Poisson statistics.
For the majority of sequences, this is approximately true, but some problematic sequences are many more mutations than expected.
To exclude these outliers, we perform a simple linear regression of the number of ``intra-clade'' mutations against time and remove sequences whose deviation from the linear fit exceeds twice the expected variance by 3.

After removing these outliers, we now bin the data by calendar week and determine the mean and standard deviation in each bin.
Evolutionary rate and putative emergence date of the variant are then estimated by weighted linear regression where each bin is weighted with the fourth root of the number of sequences in the bins.
The exact functional form of this weighing does not have a big influence on the results, but a sublinear weighing helps to counter the large variation in sequencing effort across countries and the natural imbalance due to the fact that few sequences are available early.

Due to shared ancestry, divergences of sequences are not independent data points and a regression against time is not suited to estimate evolutionary rates in general.
In particular, confidence intervals are difficult to obtain.
However, in the case of rapidly expanding variants we typically observe a large number of independent lineages emanating from one or several basal polytomies.
Along each of these lineages, mutation accumulation is independent and divergence increase allows to estimate the rate robustly.
Fig.~\ref{fig:rate_alpha} shows the increasing intra-clade divergence for clade 20I corresponding to VoC Alpha.
Both synonymous and non-synonymous within-clade average divergence increases linearly over time allowing for a robust estimate of the rate.

\begin{figure*}[tb]
    \includegraphics[width=\textwidth]{figures/rtt/20I_rtt.pdf}
    \caption{{\bf Divergence increases linearly with time (20I (Alpha)).} Each panel shows the number of within-clade mutations (total (A), amino acid changing (B), synonymous (C)) as a function of time.
    The blue line in panel A indicates the divergence cut-off, panels B\&C only show sequences that pass the divergence filter. Each panel also shows mean $\pm$ standard deviation and a weighted linear fit. Analogous figures for all clades considered are included in the appendix.
    \label{fig:rate_alpha}}
\end{figure*}

For almost all major Nextstrain clades, the average divergence increases linearly in time (similar patterns as Fig.~\ref{fig:rate_alpha}) allowing us to estimate clade specific evolutionary rates for amino acid and silent changes.
These rates are summarized in Fig.~\ref{fig:rate_summary} and Tab.~\ref{tab:rates}.

\begin{figure*}
    \includegraphics[width=\textwidth]{figures/rate_summary.pdf}
    \caption[]{{\bf Divergence and evolutionary rates of different Nextstrain clades.} Panels A,B,\&C show the estimated divergence of the founder genotype of each clade (big dot) and the subsequent divergence trend for all nucleotide changes, amino acid changes, and synonymous changes, respectively. In addition, each panel contains a regression of the divergence of clade founders vs time (gray line).
    The standard deviation expected based on Poisson statistics is indicated as shared area.
    Panel D summarizes the individual rate estimate (dots) and compares them to the estimate inter-clade rates (gray lines).
    \label{fig:rate_summary} }
\end{figure*}

Rates of synonymous change are very consistent across variants and also agree with the overall rate of synonymous changes of about 5-8 changes per genome per year, while rates of non-synonymous changes are much more variable (Fig.~\ref{fig:rate_summary}D).
Within clades, the rate of non-synonymous changes varies between 5 and 16 changes per year.
Earlier clades are estimated to have larger rates around 10-15 changes per year, while rate estimates for later clade fall  between 3 to 9 changes per year Fig.~\ref{fig:rate_progression}.
In contrast, the inter-clade non-synonymous rate exceeds 25 changes per year.

Nextstrain clades tend to be defined by long branches leading to a large polytomy.
It is thus expected that the estimated inter-clade rate exceeds the intra-clade rate purely because of this conditioning.
This effect might be particularly important early on in the pandemic where branches with as little two mutations were used to define new clades.
We thus also included composite clades 19B+, 19B++ and 20A+ containing sequences from 19A and 19B (rooted on 19B), 19A, 19B, 20A, 20B, 20C, 20D (rooted on 19B), and 20A, 20B, 20C, 20D (rooted on 20A).
The estimates for composite clades 19B+ and 20A+ are consistent with the estimates of the individual clades, while apparent rates of 19B++ are considerably higher.
The latter is due to the rapid expansion and subsequent dominance of clade 20A and its descendants \citep{korber_tracking_2020}.

\begin{table*}
\begin{tabular}{l|rrrrrr}
    \hline
    {\bf clade} &  overall rate $[y^{-1}]$ & aa rate $[y^{-1}]$ &  syn rate $[y^{-1}]$ & overall div. &  aa div. &  syn div. \\
    \hline
    19A &     19.76 &    12.79 &      6.97 &        0 &       0 &        0 \\
    19B &     19.47 &    12.37 &      7.10 &        2 &       1 &        1 \\
   19B+ &     18.55 &    11.90 &      6.65 &        2 &       1 &        1 \\
  19B++ &     28.21 &    16.37 &     11.84 &        2 &       1 &        1 \\
    20A &     15.66 &     8.40 &      7.25 &        4 &       2 &        2 \\
    20B &     14.14 &     7.61 &      6.54 &        7 &       4 &        3 \\
    20C &     16.30 &     9.52 &      6.78 &        6 &       4 &        2 \\
   20A+ &     17.15 &     9.81 &      7.33 &        4 &       2 &        2 \\
    20E &     13.65 &     5.74 &      7.91 &       11 &       4 &        7 \\
    20H &     14.50 &     8.03 &      6.47 &       19 &      15 &        4 \\
    20I &     12.18 &     5.85 &      6.33 &       28 &      18 &       10 \\
    20J &     13.31 &     6.84 &      6.47 &       33 &      21 &       12 \\
    21D &     11.36 &     6.29 &      5.07 &       25 &      12 &       13 \\
    21G &     12.05 &     6.52 &      5.53 &       26 &      18 &        8 \\
    21H &     18.17 &    10.00 &      8.17 &       29 &      20 &        9 \\
    21I &     16.33 &     8.67 &      7.66 &       28 &      23 &        5 \\
    21J &     12.25 &     6.32 &      5.93 &       32 &      27 &        5 \\
    21K &     14.37 &     6.70 &      7.67 &       54 &      45 &        9 \\
    21L &      9.02 &     3.80 &      5.22 &       66 &      49 &       17 \\
    22A &      8.10 &     2.79 &      5.31 &       69 &      51 &       18 \\
    22B &     12.07 &     4.97 &      7.11 &       64 &      49 &       15 \\
      \hline
    \end{tabular}
\caption{{\bf Evolutionary rates estimates from root-to-tip regressions for overall nucleotide changes, amino acid changes, and synonymous changes.}
    The last three columns give the distances of the clade founder sequence from the reference Wuhan/Hu-1.
\label{tab:rates}}
\end{table*}

\begin{figure}
    \includegraphics[width=0.5\textwidth]{figures/rate_progression.pdf}
    \caption{{\bf Divergence and evolutionary rates of different Nextstrain clades over time.} Synonymous rates estimates are stable in time and fluctuate around the rate estimates for between clades. Non-synonymous rate estimates are highest for clades 19A - 20C.
    \label{fig:rate_progression} }
\end{figure}

% The apparently higher non-synonymous rate suggests that at the beginning a large fraction of amino acid changes were beneficial to viral replication, not just the D614G mutation in the spike protein \citep{korber_tracking_2020}.
% This applies both to the changes within clades that were ultimately displaced by more successful variants, and inter-variant changes where the fraction of adaptive changes is even higher.



\subsection*{Diversification within clades}
A simple model for a growing variant is a super-critical branching process with growth rate $\alpha$ and an embedded mutation process.
Offspring of genomes with $i$ mutations contain $i+j$ mutations, where $j$ is a Poisson distributed number with mean $\mu t$ (mutation rate $\mu$ and generation time $t$).
The probability that offspring genomes are identical to their parents is $u = 1-e^{-\mu t}$, which for a generation time of $t=5days$ and a rate of $\mu = 15/year$ evaluates to $u=0.2$.
Since these branching process models are linear, them mean number of cases $n$ will increase exponentially with rate $\alpha$, while the number of genomes with $i$ mutations relative to the founder $m_i$ grows with rate $\alpha - u$ per generation.
\begin{equation}
    \begin{split}
        \frac{dn}{dt} &= \alpha n \\
        \frac{dm_0}{dt} &= (\alpha - u) m_0 \\
        \frac{dm_1}{dt} &= u m_0 (\alpha - u) m_1 \\
        \frac{dm_2}{dt} &= u m_1 (\alpha - u) m_2 \\
        \cdots
    \end{split}
\end{equation}
with solution $m_i = e^{(\alpha - u)t} \frac{(ut)^i}{i!}$.

For some clades, especially those that are well sampled soon after their emergence, this Poisson model is good fit to diversity accumulation and yields estimates of rates and time of origin that are compatible with the divergence regression, see Fig.~\ref{fig:within_clades}.
For other clades, especially when the founder sequence is barely observed and dominant subclades developed, such fits are not robust.
This is particularly true for the early clades 19A and 19B, as well as 21J (Delta), 21K (Omicron, BA.1), and 22B (Omicron, BA.5).

\begin{figure*}
    \includegraphics[width=\textwidth]{figures/counts/20C_counts.pdf}
    \caption{{\bf Diversification within clades.} \label{fig:within_clades}}
\end{figure*}

These Poisson weights of mutation numbers are again only valid if mutations accumulate along many independent lineages.
In particular, this assumption is violated if some lineages spread systematically faster than others either because of epidemiological factors or the mutations they carry.


\section*{Mutation tolerance}

Above, we found that the rate of synonymous mutations is comparable within and between clades without any striking signal of temporal variation.
This is expected, as synonymous positions are rarely a locus of adaptation and tend to have a small effects on fitness in large parts of the genomes of RNA viruses \citep{zanini_vivo_2017} (outside of specific regions with important RNA elements).
To assess the level of constraint on the SARS-CoV-2 genome, we made use of the ``rare mutations'' annotation by Nextclade.
Nextclade attaches each sequence to a reference tree and determines by which mutations it differs from the attachment point.
For each pango lineage (as determined by Nextclade), we count how often these ``rare mutations'' (including reversions to reference) are observed, normalize this against the relative rate of mutation away from the ancestral nucleotide (see Supp.~Fig.~\ref{fig:mutation_distribution}), and use this as a proxy for mutational tolerance.

Simply splitting mutations into those that occur in 1st, 2nd, and 3rd positions of codons already reveals strong signatures of purifying selection, see Fig.~\ref{fig:fitness_costs}.
Between 15-20\% of positions at first and second codons show almost no variation, while half of these sites are less variable then the most constrained 10\% of third positions.
The median of variation at 3rd position is more than double that at other 1st and 2nd positions.

When split by open reading frame (see Supp.~Fig.~\ref{fig:fitness_costs_supp} and \ref{fig:fitness_landscape_supp}), the most constrained are ORF1b and M, while ORF3a, ORF6, ORF7a, ORF7b, and ORF8 show little evidence of constraint, consistent with frequently observed stop-mutations in these genes.
N shows an intermediate pattern, possibly reflecting its mix of structured and unstructured regions.
Rare variation at 3rd positions is very comparable between genes.
Only E shows slightly less variation at 3rd positions than other genes with a notable dip in the middle of the gene around codon 35.



\begin{figure}
    \includegraphics[width=0.5\textwidth]{figures/fitness_cost.pdf}
    \caption{{\bf Constraints on SARS-CoV-2 mutations.}
    Almost all 3rd codon positions tolerate mutations, while 1st and 2nd positions are strongly constrained.
    About half of the 1st and 2nd codon positions are less variable than the most constrained 10\% of 3rd positions (gray line).
    \label{fig:fitness_costs}}
 \end{figure}


\section*{Discussion}
The inferred evolutionary rate of RNA viruses often decreases with the time scale across which is it estimated \citep{wertheim_purifying_2011,ghafari_prisoner_2021}.
This effect is particularly pronounced at the beginning of an outbreak following a host switch and has been attributed to methodological issues and incomplete purifying selection leading to inflated measures of diversity \citep{meyer_time_2015,ghafari_purifying_2022}.
In addition to segregating deleterious mutations, early viral evolution after a host switch will also be affect by anomalously fast adaptive evolution as a dramatic change in environment, i.e.~a host switch, likely results in many mutations that increase fitness, possibly only in epistatic combinations with others.
Such transient increases in the rate of adaptation is common in experimental evolution \citep{elena_virus_2007}.

The analysis presented here suggests that the early evolutionary rate of SARS-CoV-2 was inflated by adaptive evolution.
Since we analyzed evolution within short lived clades of SARS-CoV-2 over a span of 6 month, purifying selection has comparable efficiency for all clades.
Nevertheless, the non-synonymous evolutionary of clades circulating in late 2019 and early 2020 is about twice as high as that of clades in 2021 and 2020.

Adaptive evolution complicates phylodynamic analysis, which typically assumes that the mutation process is independent of the spread and epidemiology.
This assumption is true for neutral mutations that occur along every lineage with the same rate.
Adaptive and deleterious mutations, however, affect the ability of the virus to spread.

Deleterious mutations are relatively straightforward to account for: Strongly deleterious mutations don't spread and are only observed on terminal branches, similar to sequencing errors.
Weakly deleterious mutations can spread, but lineages that carry them tend to be short-lived.
A spectrum of mutations with different deleterious effects leads to time dependent effective evolutionary rates \citep{wertheim_purifying_2011}.
The intra-clade rates in Fig.~\ref{fig:rate_progression} are all estimated over a similar time frame and the time dependent rate phenomenon should not apply.

Adaptive evolution, however, is much harder to account for properly and can lead to more dramatic distortions.
Since the number of sites that allow beneficial mutations is small, adaptive evolution tends to be very stochastic.
Unlike neutral evolution, the rate of adaptive evolution depends on the population size and the transmission process.
Epistatic interactions can further steepen the dependence on population size and increase stochasticity.
Methods to infer emergence dates and locations of the early variants might therefore be biased or over-confident.




\bibliography{bib}
\appendix
\section*{Supplementary figures}

\begin{figure*}
    \includegraphics[width=\textwidth]{figures/mutation_distribution.pdf}
    \caption{{\bf The relative rate of different mutations in SARS-CoV-2.}
    These rates are measured from rare low frequency mutations probably subject to little purifying selection.
    \label{fig:mutation_distribution}}
\end{figure*}


\section*{Divergence summaries}

\begin{figure*}[h]
    \includegraphics[width=\textwidth]{figures/rtt/19A_rtt.pdf}
    \caption{{\bf Divergence increases linearly with time in clade 19A.}
    \label{fig:19A_divergence}}
\end{figure*}

\begin{figure*}[h]
    \includegraphics[width=\textwidth]{figures/rtt/19B_rtt.pdf}
    \caption{{\bf Divergence increases linearly with time in clade 19B.}
    \label{fig:19B_divergence}}
\end{figure*}

\begin{figure*}[h]
    \includegraphics[width=\textwidth]{figures/rtt/19B+_rtt.pdf}
    \caption{{\bf Divergence increases linearly with time in clade 19B+.}
    This figure contains sequences in clades 19 A and B rooted on clade 19B.
    \label{fig:19B+_divergence}}
\end{figure*}

\begin{figure*}[h]
    \includegraphics[width=\textwidth]{figures/rtt/19B++_rtt.pdf}
    \caption{{\bf Divergence increases linearly with time in clade 19B++.}
    This figure contains sequences in clades 19A, 19B, 20A, 20B, 20C, and 20D rooted on clade 19B.
    \label{fig:19B++_divergence}}
\end{figure*}

\begin{figure*}[h]
    \includegraphics[width=\textwidth]{figures/rtt/20A_rtt.pdf}
    \caption{{\bf Divergence increases linearly with time in clade 20A.}
    \label{fig:20A_divergence}}
\end{figure*}

\begin{figure*}[h]
    \includegraphics[width=\textwidth]{figures/rtt/20B_rtt.pdf}
    \caption{{\bf Divergence increases linearly with time in clade 20B.}
    \label{fig:20B_divergence}}
\end{figure*}

\begin{figure*}[h]
    \includegraphics[width=\textwidth]{figures/rtt/20C_rtt.pdf}
    \caption{{\bf Divergence increases linearly with time in clade 20C.}
    \label{fig:20C_divergence}}
\end{figure*}


\begin{figure*}[h]
    \includegraphics[width=\textwidth]{figures/rtt/20A+_rtt.pdf}
    \caption{{\bf Divergence increases linearly with time in clade 20A+.}
    This figure contains sequences in clades 20A,B,C,D rooted on clade 20A.
    \label{fig:20A+_divergence}}
\end{figure*}



\begin{figure*}[h]
    \includegraphics[width=\textwidth]{figures/rtt/20E_rtt.pdf}
    \caption{{\bf Divergence increases linearly with time in clade 20E.}
    \label{fig:20E_divergence}}
\end{figure*}

\begin{figure*}[h]
    \includegraphics[width=\textwidth]{figures/rtt/20H_rtt.pdf}
    \caption{{\bf Divergence increases linearly with time in clade 20H (Beta).}
    \label{fig:20H_divergence}}
\end{figure*}

\begin{figure*}[h]
    \includegraphics[width=\textwidth]{figures/rtt/20I_rtt.pdf}
    \caption{{\bf Divergence increases linearly with time in clade 20I (Alpha).}
    \label{fig:20I_divergence}}
\end{figure*}

\begin{figure*}[h]
    \includegraphics[width=\textwidth]{figures/rtt/20J_rtt.pdf}
    \caption{{\bf Divergence increases linearly with time in clade 20J (Gamma).}
    \label{fig:20J_divergence}}
\end{figure*}


\begin{figure*}[h]
    \includegraphics[width=\textwidth]{figures/rtt/21D_rtt.pdf}
    \caption{{\bf Divergence increases linearly with time in clade 21D (Eta).}
    \label{fig:21D_divergence}}
\end{figure*}

\begin{figure*}[h]
    \includegraphics[width=\textwidth]{figures/rtt/21G_rtt.pdf}
    \caption{{\bf Divergence increases linearly with time in clade 21G (Lambda).}
    \label{fig:21G_divergence}}
\end{figure*}

\begin{figure*}[h]
    \includegraphics[width=\textwidth]{figures/rtt/21H_rtt.pdf}
    \caption{{\bf Divergence increases linearly with time in clade 21H (Mu).}
    \label{fig:21H_divergence}}
\end{figure*}

\begin{figure*}[h]
    \includegraphics[width=\textwidth]{figures/rtt/21I_rtt.pdf}
    \caption{{\bf Divergence increases linearly with time in clade 21I (Delta).}
    \label{fig:21I_divergence}}
\end{figure*}

\begin{figure*}[h]
    \includegraphics[width=\textwidth]{figures/rtt/21J_rtt.pdf}
    \caption{{\bf Divergence increases linearly with time in clade 21J (Delta).}
    \label{fig:21J_divergence}}
\end{figure*}

\begin{figure*}[h]
    \includegraphics[width=\textwidth]{figures/rtt/21K_rtt.pdf}
    \caption{{\bf Divergence increases linearly with time in clade 21K (Omicron).}
    \label{fig:21K_divergence}}
\end{figure*}

\begin{figure*}[h]
    \includegraphics[width=\textwidth]{figures/rtt/21L_rtt.pdf}
    \caption{{\bf Divergence increases linearly with time in clade 21L (Omicron).}
    \label{fig:21L_divergence}}
\end{figure*}

\begin{figure*}[h]
    \includegraphics[width=\textwidth]{figures/rtt/22A_rtt.pdf}
    \caption{{\bf Divergence increases linearly with time in clade 22A (Omicron).}
    \label{fig:22A_divergence}}
\end{figure*}

\begin{figure*}[h]
    \includegraphics[width=\textwidth]{figures/rtt/22B_rtt.pdf}
    \caption{{\bf Divergence increases linearly with time in clade 22B (Omicron).}
    \label{fig:22B_divergence}}
\end{figure*}

\include{appendix_counts}
\end{document}
